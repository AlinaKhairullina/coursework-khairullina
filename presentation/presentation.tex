\documentclass[aspectratio=169, fontsize=18pt]{beamer}
\beamertemplatenavigationsymbolsempty
\usepackage[utf8]{inputenc}
\usepackage[english, russian]{babel}
\usepackage{mathptmx}
\usepackage{latexsym,amsmath,xcolor,multicol,booktabs,calligra}
\usepackage{graphicx,pstricks,listings,stackengine}

\definecolor{msublue}{RGB}{0, 56, 116}
\definecolor{msuyellow}{RGB}{167, 131, 55}
\definecolor{msugray}{RGB}{128, 128, 128}
\definecolor{msured}{RGB}{128, 0, 0}

\setbeamerfont{title}{size=\Large}
\setbeamerfont{normal text}{size=\large}
\setbeamerfont{author}{size=\large}
\setbeamerfont{institute}{size=\large}
\setbeamerfont{footline}{size=\large}


\setbeamercolor{frametitle}{bg=msublue,fg=white}
\setbeamercolor{title}{bg=msublue,fg=white}

\setbeamertemplate{footline}[frame number]

\setbeamertemplate{itemize item}{\color{msublue}\scalebox{1.5}{\raisebox{-0.1ex}{\tiny\textbullet}}}

\addtobeamertemplate{footline}{
    \ifnum\insertframenumber>1 
        \begin{picture}(0,0)
            \put(229,0){\includegraphics[width=0.6cm]{png/logo_vmk.png}}
    \end{picture}
    \fi
}


\begin{document}
\title{Исследование и разработка системы ответов к изменениям программного кода}

\author{Хайруллина Алина, ВМК МГУ\\
  }
\institute{Бакалавриат, 3 курс\\
        Научный консультант: Марков С.И.\\}

\date{19 Ноября 2024}

%1 слайд
\begin{frame}
\titlepage

    \begin{figure}[htpb]
        \begin{center}
            \includegraphics[keepaspectratio, scale=0.08]{png/logo_vmk.png}
        \end{center}
    \end{figure}
\end{frame}

%2слайд
\begin{frame}{Рецензирование кода (Code review)}

\textbf{Рецензирование кода} — обязательный процесс проверки кода другим
человеком перед его публикацией. В рецензировании участвуют все
разработчики, что повышает качество и читаемость кода, а также
обучаемость сотрудников.

\begin{block}{Вопросы у рецензента могут возникать по разным причинам:}
    \begin{itemize}
			\item {незнание определенных библиотек}
			\item различие стиля программирования
			\item отсутствие документации к коду
                \item отсутствие комментариев к коду
\end{itemize}
\end{block}
\end{frame}

%3 слайд
\begin{frame}{Примеры диалогов}

\begin{block}{ }
    \textcolor{msured}{Q:} Зачем вы заменили цикл по элементам листа для генерации форматной строчки на join()?\\
    \textcolor{msuyellow}{A:} \textcolor{msugray}{join работает эффективней, исключает полное копирование строки на каждой итерации}\\

\end{block}

\begin{block}{ }
    \textcolor{msured}{Q:} Почему удалена проверка на None после вызова process boxes?\\

    \textcolor{msuyellow}{A:} \textcolor{msugray}{Метод process boxes никогда не возвращает None, будет пустой список при отсутсвии данных}

\end{block}

\begin{block}{ }
    \textcolor{msured}{Q:} Почему raise exception вместо return exit code?\\

    \textcolor{msuyellow}{A:} \textcolor{msugray}{В данном модуле принято обрабатывать ошибки при помощи исключений}

    
\end{block}
    
\end{frame}

%4 слайд
\begin{frame}{Постановка задачи}
\begin{block}{Цель:}
    \begin{itemize}
        \item Настроить большую языковую модель \textbf{(LLM)}, которая будет отвечать на вопросы рецензентов к исходному коду с приемлемой точностью с использованием системы \textbf{Retrieval-Augmented Generation (RAG)}
    \end{itemize}
\end{block}
\begin{block}{Задачи:}
    \begin{itemize}
        \item Выбрать классы вопросов, на которые будет отвечать модель
        \item Подготовка исторических данных о проекте для использования в RAG
        \item Настройка большой языковой модели на основе собранных данных для ответа на вопросы программистов
        \item Протестировать и оценить LLM с использованием RAG
    \end{itemize}
\end{block}
\end{frame}

%5 слайд
\begin{frame}{Актуальность темы}
\begin{block}{Система автоматических ответов поможет:}
    \begin{itemize}
        \item Быстрые ответы\\
        \textcolor{msugray}{Первые эксперименты на системе code review показали, что среднее время на ответ составляет 6.5 часов, с моделью его можно сократить до 2-3 минут}
        \item Сокращение реакции при вопросе к новой версии исходного кода
        \item Исходный код быстрее попадает в репозиторий
    \end{itemize}
\end{block}
\end{frame}

\begin{frame}{План работы}
\begin{block}{ноябрь - декабрь}
    \begin{itemize}
        \item Исследование существующих решений по теме 
        \item Сбор базы данных для RAG с существующими вопросами, ответами и исправлениями
        \item Разработка функции схожести запросов в RAG
    \end{itemize}
\end{block}

\begin{block}{февраль-март}
    \begin{itemize}
        \item Зафиксировать тестовую выборку
        \item Разработать функцию оценки результатов
        \item Сбор результатов
    \end{itemize}
\end{block}

\begin{block}{апрель - май}
    \begin{itemize}
        \item Оформление текста курсовой работы
    \end{itemize}
\end{block}

\end{frame}



\end{document}
